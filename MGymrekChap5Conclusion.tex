\iffalse \bibliography{MGymrekRefs.bib} \fi

\chapter{Conclusion and future directions}
\label{chap:conc}

% Synthesis, plus propose immediate future directions
\section{Synthesis: STRs as a model for exploring the role of complex variants in human traits}
In this thesis, I have presented our work to use large scale sequencing datasets to interrogate the role of complex variants in human traits, using STRs as a model. In particular, we developed the first tools for genome-wide analysis of STRs, applied these tools to create initial population-wide catalogs of STR variation, and showed that STRs contribute on average 10-15\% of the heritablity of gene expression due to \emph{cis} variants.

These initial studies demonstrate that to fully understand how genetic variation influences phenotype, it is crucial to analyze the entire spectrum of mutations, not just those that are simplest to analyze or genotypes. There is still a tremendous amount of work to be done to improve our knowledge of the genetic control of gene regulation across the range of human cell types and ultimately how this contributes to complex traits.

The methods and results presented here pave the way toward a better understanding of the role of complex variants. Below, I discuss current challenges, new questions, and future directions toward achieving this goal.

\subsection{Methods for STR genotyping}
A major contribution of this work has been the first tool for high throughput STR genotyping from next generation sequencing data. Below I discuss extensions of this work, limitations of our current method, and potential improvements to STR genotyping going forward.

% improvements of lobstr since publication
\subsubsection{lobSTR improvements}
Since its initial development, we have continued to develop and improve the lobSTR method. We have expanded the reference to include more than one million STRs, including homopolymers. We have added support for paired-end reads, the ability to call STRs across multiple samples at once, and significant algorithmic speedups.

Recently, the author of the widely used Burrows Wheeler Aligner (BWA) \cite{LiDurbin2009a} has introduced a new tool, BWA-MEM, with significant speedups and improvements in indel sensitivity. To accomodate this bioinformatic improvement, we have added an option to lobSTR to call STRs from alignments produced by external tools. This has proven to be more sensitive and nearly as accurate as lobSTR's original sensing and alignment steps, with the added benefit of greatly reducing run times for processing large cohorts that have been previously aligned using BWA-MEM. Incorporating this capability is still a work in progress but will potentially greatly increase the sensitivity of lobSTR's alignment step and in turn the quality of genotype calls.

% limitations of our methods: 
\subsubsection{Limitations of our method}
As a result of our continued development of lobSTR and the emergence of higher coverage, PCR-free genome sequences, lobSTR can now achieve extremely high concordance upwards of 90\% with the current gold standard for STR genotyping. However, our method has several limitations:

\begin{itemize}
\item Limited sensitivity: because lobSTR only analyzes reads passing its ``sensing'' step that meet a required sequence entropy threshold, it will always face a tradeoff between run time and alignment sensitivity. The new capability of lobSTR to analyze reads aligned by alternative methods will delegate this issue to upstrem tools. However, each aligner has its own biases, and STRs embedded in low-complexity regions continue to produce problematic alignments.
\item Genotyping model limitations: lobSTR currently determines a maximum likelihood genotype at each locus using a pre-calculated model of PCR stutter noise. This model does not account for mapping or local alignment errors. Therefore, it is heavily influenced by ``outlier'' reads that have been improperly aligned. Additionally, it does not account for length-dependent biases in read coverage, sequence variations within STR alleles, or locus-specific stutter noise parameters. It will be fairly straightforward to modify the current noise model to improve some of these issues, and members of the Erlich lab are actively working on improved models.
\item Reference bias: lobSTR does not attempt to discover STR loci \emph{de novo}, but rather uses a pre-determined set of STRs discovered in the human reference genome. As a result, an STR that is truly polymorphic in the population but exhibits an allele that was too short to be recognized as an STR in the reference will not be analyzed. New tools such as the GATK HaplotypeCaller that analyze the entire genome and intregate calling of multiple types of variants have the potential to perform \emph{de novo} discovery of STRs and other complex variants.
\item Long STRs are inaccessible: the length of STRs that lobSTR can analyze is limited by the sequencing read length, since a read must entirely span an STR to be informative of the number of repeats present. With the current upper bound of read lengths around 150bp, STRs more than 100bp long are largely inaccessible. Therefore, large repeat expansions, such as those present in Huntington's Disease, Fragile X Syndrome, and dozens of other repeat expansion disorders, remain inaccessible. Although some efforts have been made \cite{DoiMonjoHoangEtAl2013}, no tool has been shown to reliably detect repeat expansions from short reads. This task is of great interest to the medical genetics community and is an active area of research in other groups.
\item Joint calling: lobSTR currently has the ability to genotype multiple samples at once, but makes limited effort to utilize information across samples to improve calls. Theoretically, features such as population allele frequencies and recurrent alignment errors across samples could inform genotype calling, as has been demonstrated in large joint calling efforts like the Exome Aggregation Consortium \cite{EXAC}.
\end{itemize}

% other methods (fungtammasan, gatk, hipstr)
\subsubsection{Recent tools for genotyping STRs from sequencing data}
Over the last several years, additional tools have arisen:

\begin{enumerate}
\item \textbf{STRViper} \cite{CaoTaskerWilladsenEtAl2014}: Uses insert size between paired end sequencing reads to detect STR variations.
\item \textbf{RepeatSeq} \cite{HighnamFranckMartinEtAl2013}: uses Bayesian model selection to genotype previously aligned STR-containing reads.
\item \textbf{STR-FM} \cite{FungtammasanAnandaHileEtAl2015}: uses a method based on lobSTR's algorithm but with a modified detection step for increased sensitivity of short repeats and may be applied to non-diploid samples.
\end{enumerate}

The first two tools operate on previously existing alignments, and so are limited by the quality of the upstream aligner. The third may provide improvements to STR calling, especially at homopolymers which are extremely noisy in sequencing data. So far, these tools have not seen widepsread use in mainstream sequence analysis.

In addition to STR-specific callers, a new class of variant callers, including GATK \cite{McKennaHannaBanksEtAl2010} HaplotypeCaller, Platypus \cite{RimmerPhanMathiesonEtAl2014}, and Scalpel \cite{NarzisiORaweIossifovEtAl2014}. These methods perform local haplotype reassembly and can incorporate sophisticated noise models, and can theoretically genotype many variant types, including STRs, quite accurately. There has so far been no systematic evaluation of their performance at STRs, but these tools may be promising for STR analysis in the future.

% long reads
\subsubsection{Long-read technology can capture long repetitive regions}
A major limitation of analyzing STRs from high throughput sequencing data is the short read length. Only reads entirely spanning an STR are informative of the repeat length, and sufficient flanking region on either side of the STR is required for accurate alignment. While the mainstream sequencing technology from Illumina is limited to sequencing at most several hundred base pairs in a single read, alternative sequencing platforms, such as PacBio's SMRT (single molecule real time) sequencing \cite{EidFehrGrayEtAl2009} and the new Nanopore technology \cite{ClarkeWuJayasingheEtAl2009}, can now produce much longer read lengths of up to several thousand base pairs.

These technologies may be used to sequence long repeats observed in expansion disorders such as Fragile X \cite{LoomisEidPelusoEtAl2012} or ataxias \cite{DoiMonjoHoangEtAl2013}, and other complex regions of the genome. Chaisson \emph{et al} \cite{ChaissonHuddlestonDennisEtAl2015} recently applied SMRT to sequence a haploid genome to 40x coverage with an average read length of 5kb. With these long reads, they were able to close 50 gaps in the human genome reference assembly, which were highly enriched for short tandem repeats and other repeats embedded in larger, more complex tandem arrays of degenerate repeats. With longer and longer read lengths, we may soon be able to analyze STRs, as well as other repetitive elements such as variable number tandem repeats (VNTRs) and retrotransposons that were previously inaccessible using sequencing studies.

\subsection{Contribution of STRs to gene expression}
We have demonstrated here that STRs make a significant contribution to variation in gene expression in lymphoblastoid cell lines. Going forward, higher powered datasets across a range of human cell types will help provide a clearer picture of individual STRs important for gene regulation and the biological mechanisms by which they act.

% limitations:
\subsubsection{Limitations of our eSTR analysis}
Our eSTR analysis faced several limitations. First, due to the low sequencing coverage and resulting low quality individual genotypes, we were underpowered to detect specific STR by gene associations with even moderate to large effect sizes. Furthermore, we showed that our analysis greatly underestimated the effects and heritability due to eSTRs. Therefore, we believe that there are many more eSTRs to discover, and that our heritability results underestimate their true contribution. 

Second, we only considered a simple linear model relating average repeat length and gene expression. Results from model organisms such as yeast \cite{GemayelVincesLegendreEtAl2010} suggest that many eSTR associations may indeed take a quadratic or other shape. Additionally, we considered only repeat length, ignoring the effects of sequence variations within STRs, which may be important for function. For instance, a recent study in Ewing Sarcoma showed that the alternate allele of a GWAS SNP combines two adjacent repeat tracts into one larger one \cite{GrunewaldBernardGilardi-HebenstreitEtAl2015}, showing the importance of STR sequence interruptions. 

Third, although we have shown strong evidence of a regulatory role for eSTRs by demonstrating enrichment near transcription start sites and enhancers and predicted ability to modulate histone modifications, the biological mechanisms by which they act remain unclear.

Finally, we found initial results that eSTRs may be important for complex traits by showing that genes implicated in autoimmune disorders by GWAS are enriched for genes regulated by an eSTR in LCLs, and that a handful of eSTRs are also associated with metabolic and other traits in an orthogonal sample. However, these analyses were again underpowered and considered eSTRs from only a single cell type (LCLs), which may not be relevant for many conditions considered. Furthermore,  we will need to replicate these results in independent cohorts and show evidence of a causal link to definitively show a role for eSTRs in complex traits.

Future studies are required to discover high confidence individual eSTRs, fully assess the contribution of eSTRs to expression heritability across a range of cell types, dissect the mechanisms by which eSTRs act, and assess the contribution of eSTRs, as well as other types of QTLs, to complex traits.

% future: bigger+better sample sizes, gtex for more tissues
\subsubsection{Higher powered datasets will provide deeper insight into the role of eSTRs}
As discusssed above, a major challenge in eQTL analysis is the small sample size of currently available datasets. As larger cohorts and deeper sequencing datasets become available, the power to detect individual eSTRs and other eQTLs will be greatly improved. For instance, the GTEx project \cite{ArdlieDelucaSegreEtAl2015} has recently generated whole genome sequencing and RNA-seq from a wide variety of tissue types across hundreds of samples. This provides an unprecedented dataset to study the effects of complex variants on gene expression and other transcription phenotypes such as splicing. 

% future: biology. including direclty predict impact of individual variation
\subsubsection{Toward biological understanding}
So far, our knowledge of biological mechanisms by which eSTRs and other eQTLs modulate transcription are largely unknown. For STRs, several hypotheses have been presented. These include altering transcription factor binding sites, changing spacing between regulatory elements, and inducing non B-DNA ``Z'' secondary structures. Additionally, intronic STRs may act by modulating alternative splicing, potentially through affecting RNA secondary structure stabilities \cite{HefferonGromanYurkEtAl2004}. We have performed preliminary analyses revealing nearly 100 intronic STRs whose repeat lengths are associated with nearby exons, suggesting this indeed may be a widespread mechanism for splicing regulation.

A key to understanding the function of non-coding variants will be the ability to pinpoint causal variants and predict their regulatory function. In our eSTR analyses, we used a novel machine learning technique, GERV \cite{ZengHashimotoKangEtAl2015a}, to demonstrate that many eSTRs are predicted to modulate certain epigenetic marks known to be associated with gene regulation. GERV is one of a handful of recently developed methods for predicting regulatory activity of non-coding regions \cite{Kelley028399,Zhou:2015aa,Alipanahi:2015aa,Lee:2015aa}. All of these methods train a machine learning model to predict an annotation of interest, such as a ChIP-sequencing dataset, based on local sequence features. These models seem to mostly capture features related to sequence specificities of transcription factors, but they can also take into account other factors including broader sequence context and co-binding of different factors. These methods have the potential to directly predict the impact of a mutation by providing a model several versions of sequences containing different alleles and quantifying the change. Moreover, since accurate models can be built using a single dataset from a cell type of interest, these methods preclude the need to measure molecular phenotypes across hundreds of samples as is required for QTL analyses. The ability to predict the impact of individual non-coding variants will likely have transformative effects on a wide range of genetics applications, including GWAS, rare variant studies, and cancer genomics.

% future: overlap with complex traits
% this is the goal
% already know examples from GWAS
% challenges
%    imputation - need large panels
%    interpretation
%    tying variants with molec mechanism driving disease
% ultimately: targets
\subsubsection{The role of expression regulation in complex traits} 
This work makes progress toward the goal of understanding the role of STRs in human traits. Several challenges remain in integrating complex non-coding regulatory variants in large scale human genetics studies:

\begin{itemize}
\item \textbf{Genotyping complex variants on a large scale}: Currently, the majority of genome-wide association studies have relied on SNP genotype arrays, rather than on whole genome sequencing. These datasets provide information on more than a million common SNP as well as CNV genotypes, and have proven useful in identifying thousands of genetic loci associated with human traits. While in most cases the true ``causal'' variants driving each signal are not directly genotyped, most common variants are in strong linkage disequilibrium with genotyped SNPs. Therefore, given a reference panel of haplotypes, the majority of common SNPs can be imputed from SNP chip data and used in association tests. STRs exhibit mutation rates often orders of magnitude higher than those of SNPs, and also experience recurrent mutations that may result in the same STR allele arising independently on multiple SNP haplotype backgrounds. As a result, STRs are less amenable than SNPs as well as less mutable complex variants to impuation. 
Despite these challenges, STRs can still appear in tight linkage disequilibrium with individual SNPs or haplotypes defined by common SNPs, and so may be imputable with at least modest accuracy. The ability to impute STRs and other variants to large previously genotyped cohorts could allow for novel association discovery in those datasets, and could explain some of the ``missing heritability'' of complex traits that cannot be explained by common SNPs. 

\item \textbf{Interpretation}: GWAS has successfully identified thousands of genetic loci associated with human traits. However, each of these associations may tag tens or hundreds of variants that are equally likely to be causally associated with the trait. A major hypothesis is that the underlying causal variants drive changes in gene expression, leading to cellular changes that result in disease. Therefore, GWAS loci should be enriched for eQTLs. While it has been shown that overall GWAS regions are enriched for eQTLs \cite{NicolaeGamazonZhangEtAl2010}, little overlap has been found between individual GWAS SNPs and eQTL loci \cite{HuangFangJostinsEtAl2015}, which may reflect the need to profile eQTLs in more specific relevant cell types. Going forward, a major challenge will be to integrate eQTLs from both SNPs and other variant types to identify the causal loci, genes, and mechanisms driving association signals in complex traits.

\end{itemize}

\section{Conclusion}
Here we have developed the first tools for analyzing STRs from next generation sequencing data and applied these tools to show that STRs make a significant contribution to variability of gene expression in humans. This work paves the way toward deeper understanding of the role of STRs and other variants in complex traits. Importantly, our results emphasize the importance of considering the entire spectrum of genetic variation in genomics studies. Studying complex variation will lead to deeper biological understanding of human disease and ultimately to therapeutic targets that may improve human health.