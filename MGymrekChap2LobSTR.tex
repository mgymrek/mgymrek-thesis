\iffalse \bibliography{MGymrekRefs.bib} \fi

\chapter{lobSTR: A short tandem repeat profiler for personal genomes}

\hzline

Most of this chapter was first published as:

\begin{itemize}
\item[] \textbf{Gymrek M}, Golan D, Rosset S, Erlich Y. lobSTR: a short tandem repeat profiler for personal genomes. \emph{Genome Research}. (2012).
\item[] \textbf{Gymrek M}, Erlich Y. Profiling short tandem repeats from short reads. Book chapter in \emph{Deep Sequencing Data Analysis} by \emph{Methods Mol Biol} (2013).
\item[] \textbf{Gymrek M}. PyBamView: a browser based application for viewing short read alignments. \emph{Bioinformatics}. (2014).
\end{itemize}

Part of this work is covered by a U.S. patent:

\begin{itemize}
\item[] Analyzing short tandem repeats from high throughput sequencing data for genetic applications. (2013). Inventors: Erlich Y, \textbf{Gymrek M}. US 2014/0163900 A1.
\end{itemize}

\hzline

\textbf{Abstract}: Short Tandem Repeats (STRs) have a wide range of applications, including medical genetics, forensics, and genetic genealogy. High throughput sequencing (HTS) has the potential to profile hundreds of thousands of STR loci. However, mainstream bioinformatics pipelines are inadequate for the task. These pipelines treat STR mapping as gapped alignment, which results in cumbersome processing times and a biased sampling of STR alleles. Here, we present lobSTR, a novel method for profiling STRs in personal genomes. lobSTR harnesses concepts from signal processing and statistical learning to avoid gapped alignment and to address the specific noise patterns in STR calling. The speed and reliability of lobSTR exceed the performance of current mainstream algorithms for STR profiling. We validated lobSTR’s accuracy by measuring its consistency in calling STRs from whole genome sequencing of two biological replicates from the same individual, by tracing Mendelian inheritance patterns in STR alleles in whole-genome sequencing of a HapMap trio, and by comparing lobSTR results to traditional molecular techniques. Encouraged by the speed and accuracy of lobSTR, we used the algorithm to conduct a comprehensive survey of STR variations in a deeply sequenced personal genome. We traced the mutation dynamics of close to 100,000 STR loci and observed more than 50,000 STR variations in a single genome. lobSTR’s implementation is an end-to-end solution. The package accepts raw sequencing reads and provides the user with the genotyping results. It is written in C/C++, includes multi-threading capabilities, and is compatible with the BAM format.

lobSTR is available at \url{https://github.com/mgymrek/lobstr-code}.

\section{Introduction}
Short tandem repeats (STRs), also known as microsatellites, are a class of genetic variations with repetitive elements of 2 to 6 nucleotides that consists of approximately a quarter million loci in the human genome \cite{Benson1999}. The repetitive structure of those loci creates unusual secondary DNA conformations that are prone to replication slippage events and result in high variability in the number of repeat elements \cite{Mirkin2007}. The spontaneous mutation rate of STRs exceeds that of any other type of known genetic variation, and can reach 1/500 mutations per locus per generation \cite{Walsh2001,BallantyneGoedbloedFangEtAl2010}, 200 fold higher than the rate of spontaneous copy number variations (CNV) \cite{Lupski2007} and 200,000 fold higher than the rate of de novo SNPs \cite{ConradKeeblerDePristoEtAl2011}.

STR variations have been instrumental in wide-ranging areas of human genetics. STR expansions are implicated in the etiology of a variety of genetic disorders, such as Huntingon’s Disease and Fragile-X Syndrome \cite{PearsonNicholEdamuraCleary2005,Mirkin2007}. Forensics DNA-fingerprinting relies on profiling autosomal STR markers and Y-chromosome STR (Y-STR) loci \cite{KayserKnijff2011}. STRs have been extensively used in genetic anthropology, where their high mutation rates create a unique capability to link recent historical events to DNA variations, including the well-known Cohen Modal Haplotype that segregates in patrilineal lines of Jewish priests \cite{SkoreckiSeligBlazerEtAl1997,ZhivotovskyUnderhillCinniogluEtAl2004}. Another relatively recent application of STR analysis is tracing cell lineages in cancer samples \cite{FrumkinWasserstromItzkovitzEtAl2008}.

Despite the plurality of applications, STR variations are not routinely analyzed in whole genome sequencing studies, mainly due to a lack of adequate tools \cite{TreangenSalzberg2012}. STRs pose a remarkable challenge to mainstream HTS analysis pipelines. First, not all reads that align to an STR locus are informative (Supplemental Figure 1A). If a single or paired-end read partially encompasses an STR locus, it provides only a lower bound on the number of repeats. Only reads that fully encompass an STR can be used for exact STR allelotyping. Second, mainstream aligners, such as BWA, generally exhibit a trade-off between run time and tolerance to insertions/deletions (indels) \cite{LiHomer2010}. Thus, profiling STR variations -- even for an expansion of three repeats in a trinucleotide STR -- would require a cumbersome gapped alignment step and lengthy processing times (Supplemental Figure 1B). Third, PCR amplification of an STR locus can create stutter noise, in which the DNA amplicons show false repeat lengths due to successive slippage events of DNA polymerase during amplification \cite{HaugeLitt1993,Ellegren2004} (Supplemental Figure 1C). Since PCR amplification is a standard step in library preparation for whole genome sequencing, an STR profiler should explicitly model and attempt to remove this noise to enhance accuracy.

Here, we present lobSTR, a rapid and accurate algorithm for STR profiling in whole genome sequencing datasets (Figure 1). Briefly, the algorithm has three steps. The first step is sensing: lobSTR swiftly scans genomic libraries, flags informative reads that fully encompass STR loci, and characterizes their STR sequence. This ab initio procedure relies on a signal processing approach that uses rapid entropy measurements to find informative STR reads followed by a Fast Fourier Transform to characterize the repeat sequence. The second step is alignment: lobSTR uses a divide and conquer strategy that anchors the non-repetitive flanking regions of STR reads to the genome to reveal the STR position and length. We use a modified reference that takes advantage of the information extracted from the sensing step to increase the alignment specificity. This step avoids a cumbersome gapped alignment and, importantly, is virtually indifferent to the magnitude of STR variations. Finally, in the third step, the pipeline allelotypes the STRs using a statistical learning approach that models the stutter noise in order to enhance the signal of the true allelic configuration. See Supplemental Material, Supplemental Figures 2-5, and Supplemental Table 1 for full details about the lobSTR algorithm.

lobSTR implementation offers a complete solution that takes raw sequencing data and reports the alleles present at each profiled STR locus. The program’s input is one or more sequencing libraries in FASTA/FASTQ or BAM format. The output is the alignment of STR reads in BAM format and the most likely alleles for each STR locus in a custom tab-delimited text format. lobSTR supports multi-threaded processing.

\section{Results}
\subsection{Comparing lobSTR to mainstream aligners}

We benchmarked lobSTR’s alignment performance with reads from an Illumina whole genome sequencing library with 101bp reads (Methods). To demonstrate its added value for STR profiling over mainstream aligners, we also ran BWA, Novoalign, and Bowtie on the same input data with and without the GATK local indel realignment tool. In addition, we ran BLAT \cite{Kent2002} to characterize STR alignment by a tool that is centered on sensitivity rather than speed. BWA and Novoalign were tested with the default parameters that can detect up to 5bp and 7bp indels, respectively. Bowtie has no indel tolerance and was evaluated as a control condition with tolerance of up to two mismatches. BLAT was tested with the default parameters that can tolerate up to 10\% divergence from the reference, which corresponds to approximately 10bp indels. To focus on the pure algorithm speed-up, all tests were executed on a single CPU.

lobSTR excelled in all the parameters required for efficient STR alignment (Table 1). First, lobSTR processed the reads 2.2 times faster than Bowtie, 22 times faster than BWA, 70 times faster than Novoalign, and almost 1000 times faster than BLAT (Figure 2A). These results indicate that there is a minimal computational payment in running lobSTR in parallel to mainstream aligners in order to augment variation calling to include STR polymorphisms. Second, as required, lobSTR reported only informative reads that fully encompass STR loci. On the other hand, the mainstream aligners reported between 2,000 to 5,000 non-informative STR reads per million input sequences, which may confound downstream calling algorithms if not removed. Third, lobSTR detected the largest number of informative reads with STR variations compared to mainstream aligners (Figure 2B). The other aligners showed a strong tendency to report STR reads with the reference allele vis-à-vis with their indel tolerance. Bowtie did not report any STR variation. After GATK local realignment, BWA and Novoalign, respectively, reported that 20\% and 25\% of the informative reads have STR variations. BLAT reported that 37\% of the informative reads have STR variations, compared to 50\% in lobSTR. Analyzing data collected from a large number of randomly ascertained STR loci \cite{PayseurJingHaasl2011} (Utah Marker Development Group) demonstrates that 33-66\% of STR sequence reads should exhibit a non-reference allele (see Methods). This suggests that lobSTR’s results are more representative of the true rate of STR variations than mainstream alignment tools.

Reporting STR reads with non-reference alleles is crucial for profiling pathogenic mutations. We further explored whether lobSTR can correctly detect disease alleles of dominant trinucleotide repeat expansion disorders. As test cases, we focused on two conditions that can be theoretically profiled using standard Illumina runs. The first condition was a GCN expansion in PABPN1 that causes oculopharyngeal muscular dystrophy (OPMD) \cite{BraisBouchardXieEtAl1998}, where the normal allele exhibits 10 repeats and the pathogenic allele spectrum for the dominant form is between 12 to 17 repeats \cite{PearsonNicholEdamuraCleary2005}. The second condition was a GCG expansion in HOXD13 that is implicated in synpolydactyly \cite{MuragakiMundlosUptonEtAl1996}, a severe limb malformation, where the normal allele is 15 repeats and the documented pathogenic allele spectrum is between 22-29 repeats \cite{PearsonNicholEdamuraCleary2005}. To simulate each condition, we generated 100 reads of length 101bp that were equally sampled from the disease locus consisting of a normal and pathogenic allele with 100bp flanking upstream and downstream regions with 1\% sequencing error rate. For both simulated disease conditions, lobSTR accurately aligned the normal and pathogenic reads to the correct location in the genome. All aligned reads were informative and the allelotyping step correctly assigned a heterozygous state to the disease loci with the correct repeat lengths: (10,15) for PABPN1 and (15, 22) for HOXD13. In stark contrast, BWA failed to correctly align reads from the pathogenic alleles of both loci. Only reference reads were reported (Figure 2C).

\subsection{Measuring lobSTR concordance using biological replicates}
To explore the precision of lobSTR, we conducted genome-wide STR profiling of blood and saliva samples from the same individual \cite{LamClarkChenEtAl2012}. These samples were sequenced using Illumina HiSeq2000 with 101bp PE to a mean autosomal coverage of 50x and 102x, respectively. lobSTR ran with default parameters on 20 CPUs and analyzed the two datasets within 12 and 22 hours respectively. After filtering loci with low quality calls, 143,793 shared STRs were covered in the two datasets with at least one read and 79,771 STRs were covered with 10 reads or more (Figure 3A). 

We quantified the rate of discordant autosomal calls between the two samples. We focused on two measurements: the genotype discordance rate and the allelic discordance rate \cite{PompanonBoninBellemainEtAl2005}. The former reports as an error any mismatch between corresponding calls, whereas the latter reports only the fraction of discordant alleles in corresponding calls. For example, consider a locus that is called (A,B) in the saliva sample and (A,C) in the blood sample. This locus shows a single genotype discordance, but only 0.5 allelic discordance, since the A allele was correct. 

Both types of error greatly diminished with sufficient coverage (Figure 3B,C). At 5x coverage, the genotype discordance rate was 11\% and the allelotype discordance was 5\%; At 21x coverage, the genotype discordance rate was 3\% and the allelotype discordance rate was 2\%. Similar to STR studies with capillary platforms \cite{WeberBroman2001}, most of the errors were generated in dinucleotide STR loci, whereas other types of STRs showed moderate and similar error rates. The dinucleotide error rates presumably stem from two factors: first, these loci usually show the highest heterozygosity rates \cite{ChakrabortyKimmelStiversEtAl1997,BrinkmannKlintscharNeuhuberEtAl1998,PembertonSandefurJakobssonEtAl2009}. Therefore, they require on average more sequence reads to be correctly called. Second, dinucleotide STRs are more prone to stutter noise \cite{Ellegren2004} and their higher error rates might be due to residual noise after lobSTR stutter deconvlution.

We further analyzed the STR length differences in discordant calls. To avoid analyzing errors that are simply due to allele drop-outs, we focused on discordant calls that were both heterozygous in blood and saliva. At a coverage of ≥5x, more than 90\% of the errors showed a single repeat unit difference and 99\% of the errors were within two repeat units (Figure 3D). This indicates that incorrect alignment of STRs has a minimal effect on allelotyping results, and that stutter is likely the main source of noise. We also found that only 0.8\% of calls at heterozygous loci showed a difference due to an incomplete repeat unit. This highlights that lobSTR can determine STR alleles at a single base-pair resolution. 

\subsection{Tracing Mendelian inheritance using lobSTR}
To further explore lobSTR performance, we conducted a genome-wide STR profiling of a HapMap trio -- a father (NA12877), mother (NA12878), and son (NA12882) -- from the CEU population that were sequenced using 100PE reads on a HiSeq2000 (Table 2). The average autosomal coverage was 50x and average STR coverage was 14x. At ≥10x coverage threshold, 57\% of the STRs in the CEU trio had a non-reference allele. 

In general, deviations of offspring’s STR alleles from Mendelian inheritance (MI) indicate a potential calling error \cite{EwenBahloTreloarEtAl2000}. With 5x coverage across all trio members, the MI rate was 95\%; with 10x coverage, MI was 97\%; and with coverage threshold of 15 or more, MI was 99\%. (Figure 4A). We also repeated the analysis only with discordant parental sites (for example, A/B call in one parent and A/C call in another parent). We noticed a drop to 93\% in the MI patterns with a low coverage threshold of 5x, which is mainly because of partial coverage of heterozygous sites in the parents. The MI rate was recovered to the same level with higher coverage threshold. At 17x coverage 99\% of sites showed a perfect Mendelian segregation pattern (Figure 4B). 

\subsection{Validating lobSTR accuracy with DNA electrophoresis}
We sought to compare lobSTR calls to the results of DNA electrophoresis, which is considered the gold standard for STR allelotyping. First, we focused on a set of STR markers that are used for forensic DNA fingerprinting. As an input for lobSTR, we sequenced a male genome from our lab collection with three runs of Illumina GAIIx for 101PE cycles that yielded ~740 million reads. The autosomal sequencing coverage was 36x according to alignment with mainstream algorithms. lobSTR identified 1.6 million informative reads that mapped to ~140,000 STR loci, with an average of 4.91x coverage of diploid STR loci. In parallel, we used a commercial forensic kit to genotype 14 autosomal STR markers on a capillary electrophoresis platform. Thirteen out of 14 markers were covered by at least a single sequence read and 8 markers were covered by at least 3 sequence reads. The marker that was not covered spanned more than 129bp, exceeding the limit for detecting informative reads with the 101bp sequence reads.

We observed good concordance between lobSTR and the capillary results (Table 3). lobSTR correctly called all but one of the 8 markers that were covered by at least 3 reads and most of the alleles in loci that were covered with 2 or less reads. Remarkably, some of these markers, such as D8S1179, displayed two heterozygous alleles that did not match the reference. Other alleles, such as in Penta D and Penta E, correctly returned 20bp and 25bp length differences from the reference allele, respectively. The capillary results of one tetranucleotide marker, THO1, exhibited a non-integer number of copies (9 repeats + 3bp). lobSTR reported exactly the same results, further demonstrating that STRs can be called within a single base pair resolution. lobSTR also correctly called a homozygous STR that was covered by a single read. In another 4 markers with coverage of ≤2x, lobSTR correctly called one allele and missed the other allele due to sequencing coverage. We observed only a single erroneous call due to stutter noise in the D5S818 locus. This homozygous locus was covered by three sequence reads: two correct and one with a single repeat expansion. With such a low sequencing coverage, the allelotyping algorithm was not able to identify the noisy read and assigned a heterozygous state to the locus. 

Next, we evaluated lobSTR calls made in 12 low-pass sequenced genomes from the Human Genome Diversity Project (HGDP) \cite{GreenKrauseBriggsEtAl2010,ReichGreenKircherEtAl2010}. Five genomes had coverage of 1.4x-1.9x with 109bp reads, and the other seven had coverage of 4.8x-7.7x with 77bp reads (Supplemental Table 3). One hundred and ninety five STRs with equivalent entries in the lobSTR reference have been genotyped in these genomes using DNA electrophoresis as part of the CEPH-HGDP panel \cite{RamachandranDeshpandeRosemanEtAl2005,PembertonSandefurJakobssonEtAl2009}. Combining lobSTR results from all datasets gave 59 comparable markers with coverage of 3-5 reads with a median coverage of 3x (Supplemental Table 4). Despite the low coverage, lobSTR correctly returned 75\% of the genotypes and 85\% of the allele calls. Most of the alleles showed at least 5bp difference from the reference and some alleles showed a difference of 24bp and were correctly called. We did not observe a significant correlation between errors and the size of the variation. 

\subsection{Genome-wide STR profiling confirms previously locus-centric observations}
Encouraged by the accuracy and speed of lobSTR, we harnessed our pipeline to establish a reliable reference for future studies. Our input dataset was a male individual that, as of today, has been sequenced to highest coverage of 126-fold from a blood sample \cite{AjayParkerAbaanEtAl2011}. Fourteen billion sequencing reads were obtained from 100bp PE runs on Illumina GAIIx and HiSeq 2000. lobSTR ran for 26 hours using 25 CPUs. It aligned ~6 million reads to approximately 180,000 STR loci out of the 249,000 in the Tandem Repeat Table reference with average coverage 20.82 for autosomal loci. The average reference allele length of undetected loci was 150bp, whereas the mean reference length of detected loci was 41bp. Therefore, in most cases, the undetected loci could not physically be spanned by a single read of the current sequencing length.

We assigned each autosomal STR to one of four allelotype categories: both alleles match the reference (homozygous reference), one allele matches the reference (heterozygous reference), both alleles do not match the reference but are the same (homozygous non-reference), and both alleles are different and do not match the reference (heterozygous non-reference). In all previous experiments, a coverage threshold of 20x resulted with near-perfect STR calling even for dinucleotide loci. To increase the reliability of our results, we focus the analysis on the 97,844 loci that were called with at least this sequencing coverage. The length distribution of these alleles in the reference was mainly between 25-50bp with a low number of very long STRs (Figure 5A). 

Similar to the other genomes in this study, 55\% (52,338) of the STR loci differed from the reference: 22,271 (23\%) loci were heterozygous reference, 15,515 (16\%) loci were homozygous non-reference, and 14,552 (15\%) loci were heterozygous non-reference. The other 43,335 (45\%) loci were homozygous reference. Some of the variations reached to 49bp difference from the reference allele. On average, STR variations showed 6.3bp difference from the reference allele and 41\% of the variations were more than 5bp away from the reference (Figure 5B). Thus, mainstream-dependent analysis pipelines that can tolerate only a few nucleotide indels, such as BWA, are likely to miss most STR variations.

The genome-wide STR dynamics reported by lobSTR confirm previous findings of locus-centric studies. The rate of STR polymorphism showed a striking correlation with the repeat unit length (Figure 5C). Dinucleotide STRs are nearly equally likely to fall into any of the above four categories, whereas hexanucleotide STRs are most likely to match the reference. This trend matches results of a previous study that measured the mutation rate of a few hundreds of Y-STR loci as a function of repeat unit length \cite{JarveZhivotovskyRootsiEtAl2009}.  Similar to our results, the authors showed that penta- and hexanucleotide repeats mutate at half the rate of tri- and tetra-nucleotide repeats. We also found that the rate of STR polymorphism is significantly correlated to the length of the STR allele in the reference (Figure 5D). The non-reference loci (n=52,338) had significantly greater lengths than loci that are homozygous reference (n=43,335) (p<0.05, one-sided Mann-Whitney test for each allelotype category versus reference) as previously reported in studies that analyzed a few dozen STRs \cite{BrinkmannKlintscharNeuhuberEtAl1998,Ellegren2000}. 

We also used lobSTR to determine genome-wide trends of STRs at single base pair resolution (Figure 5E). Overall, 99\% of alleles varying from the reference allele showed differences that were complete multiples of the STR unit. This trend varied by period, with dinucleotide STRs least likely (0.3\%) to differ by an incomplete motif unit and hexanucleotide most likely (4.7\%).

Finally, lobSTR reported significant differences between repeat variations in intronic and exonic regions (Figure 5F). Intronic trinucleotide STRs were twice as likely to exhibit at least one non-reference allele than exonic regions (0.480-0.502 95\% CI and 0.179-0.336 95\% CI for introns and exons, respectively), and nearly five times as likely to exhibit two non-reference alleles (0.107-0.119 95\% CI and 0-0.047 95\% CI for introns and exons, respectively). Significantly, lobSTR reported that 1.9\% (62 out of 3276) of the intronic trinucleotide STRs showed length differences that were not a multiple of three nucleotides. On the other hand, all reported exonic trinucleotide variants retained the reading frame. In addition, lobSTR allelotyped 34,667 intronic and 7 exonic non-trinucleotide STRs. Of the intronic non-trinucleotide STRs, 18,277 (53\%) showed at least one allele with a frameshift deviation, and 8,686 (25\%) showed two frameshifted alleles. Surprisingly, 3 of the 7 exonic loci, all tetranucleotides, showed expansions by units of 4bp, which would result in a frameshift mutation. In one case, in exon 8 of DCHS2, the frameshift variation was homozygous. This call was supported by 33 independent reads, showing a potential loss of function in this gene.

Taken together, the overall findings of lobSTR in this genome serve as a biological validation for the accuracy and utility of genome-wide STR profiling using our technique.

\section{Discussion}
STR profiling techniques have changed very little in the past two decades, relying on the faithful yet cumbersome capillary electrophoresis technique to scan a few dozen loci at a time. The advent of HTS has ushered in the opportunity to conduct genome-wide STR variation analyses. Here, we presented an end-to-end solution for this task. Our solution bypasses the gapped alignment problem, has no inherent indel limitation, and can reliably profile highly polymorphic STRs at a single base pair resolution. We provided a detailed comparison between lobSTR and popular mainstream aligners and showed that even with long reads, these aligners are significantly biased towards the detection of the reference allele. We have established the feasibility of lobSTR to profile STR loci from a total of 20 genomic datasets and demonstrated the strategy’s accuracy by analyzing its consistency, ability to trace Mendelian inheritance, and comparing its results to orthogonal molecular techniques. Moreover, our genome-wide STR analysis confirms previous biological observations, which further highlights the algorithmic validity. 

lobSTR results from the trio genomes and the Ajay et al. genome consistently showed genome-wide polymorphism rates of 55\%-57\% for STRs with lengths 25bp and over. A recent study by McIver et al. \cite{McIverFondonSkinnerEtAl2011} evaluated the performance of STR calling using post-BWA alignment files with a set of quality rules. Using a mixture of Illumina 45-100bp reads and 454 reads from two trios in the 1000Genoems project, they reported that 1.1\% of the STRs with lengths of 20bp and over were polymorphic. We wondered if the polymorphism discrepancy between the studies could be explained by the shorter reading lengths in the McIver study that biased their calls to very short, less polymorphic STRs. However, when we ran lobSTR on the 1000Genomes CEU trio datasets (Methods), we found again that 57\% of the STRs were polymorphic (25,885 out of 45,461 STRs that were called with ≥5x coverage at the three genomes). These results suggest that STR profiling that is restricted by the default BWA indel tolerance -- 5bp for the Illumina datasets in the McIver et al. study -- can significantly reduce the sensitivity for observing STR variations. 

We envision that lobSTR will be used in parallel to conventional analysis pipelines in order to augment variation calling to STR loci. The fast running time of our algorithm should not impose a significant computational burden on users. A low coverage genome of 5x takes about an hour on a standard server with 25 CPUs, a high coverage genome of 30x takes eight hours using the same settings, and a ultra-high covered genome of 126x takes 26 hours (Supplemental Table 2).

Currently, the major barrier for STR profiling is the sequencing read length, as the number of detectable STRs is limited to those that are entirely spanned by a single read. To test the effect of genomic coverage on STR profiling, we sampled reads from the 126x genome and calculated the amount of reported STRs (Supplemental Figure 6). With genome-wide coverage of 40x, there are more than 100,000 STRs that will be pass an STR-coverage threshold of 10x. However, higher genomic coverage does not linearly improve the number of STRs that pass this threshold, marking a potential upper bound of sequencing read lengths of 100bp. We also explored the utility of the longer reads by Sanger, 454, and IonTorrent for STR profiling of personal genomes using lobSTR (Supplemental Table 5 and Supplemental Text). Longer reads indeed increased the number of reported STR loci compared to the same autosomal coverage by Illumina. However, out of these, Sanger seemed to be the only method to produce reliable STR reads. We expect that as sequencing reads continue to increase in both length and quality, lobSTR’s performance will further improve and allow inclusion of a larger number of STR variations. Ultimately, these will include large pathogenic expansion, such as those in Huntington’s disease, which can span more than 100bp. 

As of today, sequence analysis algorithms can detect almost any type of genetic variations, from SNPs \cite{GoyaSunMorinEtAl2010} and indels \cite{KoboldtChenWylieEtAl2009,GoyaSunMorinEtAl2010} to CNVs and chromosomal translocations \cite{ChenWallisMcLellanEtAl2009}. lobSTR adds a new layer of information with tens of thousands of highly polymorphic genetic variations that have a multitude of applications, from personal genomics, to population studies, forensics analysis, and cancer genome profiling.

\section{Methods}
\subsection{Comparing lobSTR to mainstream aligners}
All alignment strategies were tested in a Linux environment, on a server with 4 twelve-core AMD Opteron 6100 and 128Gbyte of RAM. The following software versions were tested: BWA version 0.5.7, Bowtie version 0.12.7, and Novoalign freeware version 2.7.13, BLAT version 34, and GATK version 1.3-21.

The input was 5 million Illumina reads of the male sample from our lab collection. BLAT results were filtered to include only the top hit for each read. We suppressed multi-mappers in all other tools. Informative STR reads were identified by the intersectBed tool of the Bedtools packages \cite{QuinlanHall2010}. We converted CIGAR scores to the number of base pairs difference from the reference allele by counting any insertions or deletions falling within and directly adjacent to the STR region. Simulating reads from pathogenic STR loci was conducted using a simple Python script available by request from the authors. 

\subsection{Determining the expected number of non-reference reads}

A previous study by The Utah Marker Development Group has shown that 70\% of thousands of randomly chosen tetranucleotide STR loci are polymorphic. We also re-analyzed Payseur et al. data to infer the polymorphism rate in STRs with length ≥25bp in the assembled genome of Craig Venter using results reported in their Supplementary Tables 1-5. Concordant with the Utah study, this rate was 66\%.

The rate of non-reference STR reads is bounded between two extreme cases. The lower bound is that all polymorphic STRs are heterozygous with a reference allele. Thus, only half of the reads from variable loci will show a non-reference allele, which gives 33\% as a lower bound. The upper bound is that all polymorphic STRs are different from the reference in their two alleles. In this case, every read from a variable locus will show a non-reference allele, which gives 66\% as an upper bound.  

\subsection{Biological replicates analysis}
Raw reads for blood-derived and saliva-derived genomic DNA from the same individual were downloaded from the NCBI short read archive (\url{www.ncbi.nlm.nih.gov/sra}) with accessions SRX097307 and SRX097312, respectively. Loci in which (1) less than 75\% of reads agreed with the allelotype call in both samples or (2) the locus was covered in either sample by more than three times the mean coverage level were removed from the analysis.

\subsection{CEU trio data for Mendelian inheritance}
The HapMap CEU trio were NA12877 (father), NA12878 (mother), and NA12882 (son). Raw reads were downloaded from the European short read archive (\url{www.ebi.ac.uk/ena/}) with accessions ERP001228, ERP001229, and ERP001230, respectively. To determine if an STR followed Mendelian inheritance, we required that the alleles detected in the son could be explained by inheriting one allele from each parent. Low quality loci were filtered as described in Biological replicates analysis.

\subsection{Validating lobSTR accuracy using capillary electrophoresis}
Four Catch-All buccal swabs (Epicenter, QEC89100) were used to collect the DNA sample according to the manufacturer protocol. gDNA was extracted by QuickExtract (Epicentre), followed by phenol-chloroform purification and ethanol precipitation. Library preparation was performed according to the standard Illumina protocol. Three runs of 101bp paired-end with a GAIIx platform were used for sequencing. The study was approved by MIT’s Committee on the Use of Humans as Experimental Subjects (COUHES). The general sequencing coverage was analyzed as previously reported \cite{ErlichEdvardsonHodgesEtAl2011}. 

Capillary electrophoresis results were obtained from Sorenson Genomics laboratory using the commercial Promega PowerPlex 16 system. To find the genomic positions of these loci, we downloaded corresponding primers that target these loci from the Short Tandem Repeats Internet Database (STRBASE) website (\url{http://www.cstl.nist.gov/strbase/}) of the US National Institute of Standards and Technology (NIST) and used the In Silico PCR tool on the UCSC genome browser to reveal their location. Two loci had proprietary primers and their genomic location could not be identified. The STR repeats in the sequencing file were converted to the PowerPlex allele nomenclature using the NIST definitions.

\subsection{Obtaining CEPH-HGDP STR allelotypes}
STR allelotypes along with a table of RefSeq reference alleles were downloaded from the Rosenberg lab site (\url{www.stanford.edu/group/rosenberglab/repeatsDownload.html}). The allelotypes were given as the number of repeats converted from PCR product size as described in Pemberton et al. \cite{PembertonSandefurJakobssonEtAl2009}. The repeat number is given as reference repeat number plus the difference in product size from the reference divided by the motif size. Sequence data were downloaded from the NCBI Short Read Archive with accessions: ERX004003, ERX004002, ERX004001, ERX004000, ERX0039999, ERX004007, ERX007978, ERX007977, ERX007976, ERX007975, ERX007974, ERX007973, ERX007972.

Using the STS marker table available from the UCSC Genome Browser, we converted the Pemberton et al. markers to hg18 genomic coordinates and annotated them using the TRF table. lobSTR calls that are supported by three or more reads were converted to the Pemberton results. Non-integer repeats reported by lobSTR were rounded to the smallest integer for compatibility with Pemberton data. Markers that could not be faithfully annotated were removed from the analysis.

\subsection{Genome-wide STR profiling of a deeply sequenced personal genome}
Raw sequencing reads for accession number ERP000765 were downloaded from the European Nucleotide Archive’s Sequence Read Archive (\url{http://www.ebi.ac.uk/ena/}). The Mann-Whitney test was performed using the wilcox.test function in R. Confidence intervals were calculated using a normal approximation to the Poisson distribution, with a 95\% confidence interval of $\lambda\pm 1.96\sqrt{\lambda}$, where $\lambda$ is the estimated mean of the distribution. Only loci with greater than 20-fold coverage were included in the analysis. Exon and intron coordinates were obtained from the UCSC table browser for human genome build hg18. 

\subsection{1000 Genomes data analysis for the McIver Study}
The HapMap CEU trio were NA12878 (daughter), NA12891 (father), and NA12892 (mother). Raw sequencing reads for the CEU HapMap trios with length of at least 47bp were downloaded from the 1000 genomes NCBI ftp site (\url{ftp://ftp-trace.ncbi.nih.gov/1000genomes/ftp/}). 228, 274, and 214 files were included for individuals NA12878, NA12891, and NA12892. To accommodate the shorter read lengths, lobSTR was run with non-default parameters –fft-window-size 20 –fft-window-step 10, --maxflank 100, and –extend-flank 5. 

\section{Acknowledgements}
Y.E is an Andria and Paul Heafy Family Fellow. This publication was supported by the National Defense Science and Engineering Graduate Fellowship (M.G.) and by a fellowship from the Edmond J. Safra Bioinformatics program at Tel-Aviv University (D.G.). D.G. and S.R. acknowledge support from Israeli Science Foundation grant ISF 1227/09 and an IBM Open Collaborative Research grant. We thank Mona Sheikh, Dina Esposito, and Alon Goren for useful comments on the manuscript, Assaf Gordon for his assistance with multithreading programming, Cole Trapnell for his assistance with preparing lobSTR executables, Mona Sheikh and Sam Sinai for testing lobSTR code, and Dina Esposito for preparing samples for genotyping. 


%% TODO figures/tables and legends